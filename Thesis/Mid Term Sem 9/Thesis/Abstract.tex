\vspace*{0.5mm}
\begin{center}
\large{Abstract}
\end{center}

\vspace*{1cm}
\begin{flushleft}
Title: ENERGY-EFFICIENT NEUROMORPHIC COMPUTING
\end{flushleft}

\begin{flushleft}
By Khritish Kumar Behera
\end{flushleft}

\begin{flushleft}
A dissertation submitted in partial fulfillment of the requirements for the degree of Master of Science at Indian Institute of Science Education and Research Bhopal 
\end{flushleft}

\begin{center}
Indian Institute of Science Education and Research Bhopal, Year.
\end{center}

\begin{center}
Advisor: Dr. Kuntal Roy, Electrical Engineering and Computer Sceince (EECS) Department
\end{center}

\vspace*{1cm}
\doublespacing
Building Energy-Efficient Neuromorphic Computing device, to perform neuromorphic computation more efficiently in terms of energy consumption. Transitioning from transistor based devices to spintronics based devices because of inherent advantages of spintronics devices over transistor ones. Sintronic devices are non-volatile in nature, they can process, communicate and store information. Whereas in case of transistor based devices it requires extra circuitry to have the non-volatile nature.
We have gone far faster in terms of computation than our human brain. But biological systems have some distinct capabilities like recognizing faces from the crowd, even when wearing a mask. Biological systems require very little power to operate. The ultimate solution for low-power brain-inspired computing is to replicate the neural architecture of the brain on a chip. The aim of the project is to design and fabricate neuromorphic spin-devices using multiferroics exploiting its novel spintronic device concepts. As a proof-of-concept we plan to implement a character recognition system using back-propagation neural network algorithm.
Spintronics based device are endurable to cosmic waves, which makes it suitable for outer space instruments. In the future the spintronics based device will replace the transistor based device.
